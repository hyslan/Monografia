\chapter{Espaços Vetoriais}
Começaremos pela definição de um espaço vetorial, onde, podemos tratar como um vetor ao designar um elemento do espaço vetorial de um número $\mathbb{R}$ definido tal que:

\noindent\textbf{Definição 01:} Seja um conjunto V, não vazio, com duas operações: soma, $V \times V \rightarrow V$, e multiplicação por escalar, $R \times V \rightarrow V$, tais que, para quaisquer $u, v, w \in \mathbb{R}$, satisfaçam as propriedades: \nocite{boldrini1980}
\begin{enumerate}
	\item $(u + v) + w = u + (v + w), \forall$ $u, v, w \in V$ (propriedade associativa.) 
	\item $1u = u$.
	\item $u + v = v + u, \forall$ $u, v \in V$ (propriedade comutativa).
	\item $\exists$ $0$ $\in V$ tal que $u + 0 = u$.
	\item $\exists$ $-u \in V$ tal que $u + (-u) = 0$.
	\item $a(u + v) = au + av$.
	\item $(a + b)v = av + bv$.
	\item $(ab)v = a(bv)$.
	\item $1u = u$.
\end{enumerate}

\noindent\textbf{Observação:} $\textbf{0}$ é o vetor nulo. \nocite{ulhoa2018}

\noindent\textbf{Observação:} Limitaremos nossa discussão, demonstrações e aplicações dentro do conjunto dos números reais apenas.

\noindent\textbf{Exemplo 01:} Suponhamos uma matriz $M_{(2, 2)}$, onde, é denotado por $M_{(m,n)}$, dado por $M = [a_{ij}]_{m \times n}$ podendo ser interpretada dessa forma, $V = M_{(2, 2)}$, onde $V$, é um conjunto não vazio, seu escalar pertencente ao conjunto dos $\mathbb{R}$, que satisfazem todas as propriedades de um espaço vetorial.

\begin{figure}[H]
	\centering
	\includegraphics[scale=1.0]{exemplo01.png}
	\caption{Exemplo 01: Vetor no plano.}
\end{figure}

A partir disto, podemos perceber o uso analítico dos espaços vetoriais para resolução de problemas em geral. Vejamos mais alguns exemplos.

\noindent\textbf{Exemplo 02:} O exemplo anterior, trata-se de uma matriz de $\mathbb{R}^2$ pode ser dito como, no plano, agora iremos expandir para $\mathbb{R}^3$, seja um vetor $A = (x, y, z)$ ou representado pela forma matricial:

\[
A = \begin{bmatrix}
	a \\ b \\ c
\end{bmatrix}
\]
\noindent Assim, por quaisquer números reais, podemos fazer uma projeção ortogonal no espaço, segue um exemplo traçado:

\begin{figure}[H]
	\centering
	\includegraphics[scale=0.30]{exemplo02.png}
	\caption{Exemplo 02: Exemplo de vetor no espaço.}
\end{figure}

\noindent\textbf{Exemplo 03:} Consideremos $n-uplas$ de números reais.

$V = \mathbb{R}^n = \{(x_1, x_2, \ldots, x_n); x_i \in \mathbb{R}\}$

e se $u = (x_1, x_2, \ldots, x_n), v = (y_1, y_2, \ldots, y_n)$ e $a \in \mathbb{R}$,

$u + v = (x_1 + y_1, x_2, y_2, \ldots, x_n, y_n)$ e $au = (ax_1, ax_2, \ldots, ax_n)$

Por tratarmos de uma quantidade $n$ de números, o campo tridimensional deixa de ser visto, e passamos a ter $\mathbb{R}^n$ dimensões, as propriedades não deixam de valer independente a quantidade de dimensões.

% SubEspaço vetorial
\section{Subespaços Vetoriais}
Nesta seção iremos introduzir conceitos no estudo de espaço vetorial para subespaço vetorial.

\noindent\textbf{Definição 02:} Dado um espaço vetorial V, um subconjunto W, não vazio, será um subespaço vetorial de V se:
\begin{enumerate}
	\item Para quaisquer $u, v \in W$ tivermos $u + v \in W$.
	\item Para quaisquer $a \in R, u \in W$ tivermos $au \in W$.
	\end{enumerate}

\noindent\textbf{Teorema 01:} Um subconjunto não vazio $W$ de $V$ é um subespaço de $V$ se, e somente se, para cada par de vetores $\alpha, \beta$ em $W$ e cada escalar $c$ em $F$, o vetor $c\alpha + \beta$ está em $W$.

\noindent\textbf{Demonstração:} Suponhamos que $W$ seja um subconjunto não vazio de $V$, tal que, $c\alpha + \beta$ pertença a $W$ para todos os vetores $\alpha$, $\beta$ em $W$ e todos escalares $c$ em $F$. Como $W$ é não vazio, existe um vetor $\rho$ em $W$, logo $(-1) \rho + \rho = 0$ está em $W$. Então se $\alpha$ é um vetor arbitrário em $W$ e $c$ é um escalar arbitrário, o vetor $c\alpha = c\alpha + 0$ está em $W$. Em particular $(-l)\alpha = -\alpha$ está em $W$. Finalmente se $\alpha$ e $\beta$ estão em $W$, então $\alpha + \beta = 1\alpha + \beta$ está em $W$.
Assim, $W$ é um subespaço de $V$. \nocite{hoffman1979}

\noindent\textbf{Exemplo 04:} Considere o espaço vetorial $\mathbb{R}^3$. O conjunto de todos os vetores que residem no plano $xy$, ou seja, $\{(x, y, 0) \mid x, y \in \mathbb{R}\}$, forma um subespaço vetorial de $\mathbb{R}^3$.

Se o conjunto dado forma um subespaço vetorial de $\mathbb{R}^3$, precisamos verificar as três propriedades fundamentais:

\begin{enumerate}
    \item Contém o vetor nulo: O vetor nulo em $\mathbb{R}^3$ é $(0,0,0)$. Este vetor também está contido no plano $xy$, pois $z = 0$.
    
    \item É fechado sob adição: Se tomarmos dois vetores $(x_1, y_1, 0)$ e $(x_2, y_2, 0)$ no plano $xy$, a sua soma será $(x_1 + x_2, y_1 + y_2, 0)$, que também reside no plano $xy$.
    
    \item É fechado sob multiplicação por escalar: Para qualquer escalar $c$ e vetor $(x, y, 0)$ no plano $xy$, $c \cdot (x, y, 0) = (cx, cy, 0)$, que também está no plano $xy$.
\end{enumerate}

Então, o conjunto de todos os vetores $(x, y, 0)$ com $x, y \in \mathbb{R}$ forma um subespaço vetorial de $\mathbb{R}^3$.

\noindent\textbf{Exemplo 05:} No espaço vetorial das funções reais de uma variável real, $V = \{f(x) \mid f: \mathbb{R} \rightarrow \mathbb{R}\}$, considere o conjunto de todas as funções lineares, ou seja, $\{f(x) = mx + b \mid m, b \in \mathbb{R}\}$. Esse conjunto forma um subespaço vetorial de $V$. Novamente, você pode verificar as propriedades para confirmar.

Se o conjunto dado forma um subespaço vetorial de $V$, novamente precisamos verificar as três propriedades fundamentais:

\begin{enumerate}
    \item Contém a função nula: A função nula em $V$ é $f(x) = 0$. Esta função é uma função linear, pois pode ser escrita como $f(x) = 0 \cdot x + 0$. Portanto, a função nula está contida no conjunto.
    
    \item É fechado sob adição: Se tomarmos duas funções lineares $f_1(x) = m_1x + b_1$ e $f_2(x) = m_2x + b_2$, a sua soma será $f_1(x) + f_2(x) = (m_1 + m_2)x + (b_1 + b_2)$, que também é uma função linear. Portanto, o conjunto é fechado sob adição.
    
    \item É fechado sob multiplicação por escalar: Para qualquer escalar $c$ e função linear $f(x) = mx + b$, a multiplicação por escalar $cf(x) = c(mx + b) = (cm)x + (cb)$ também é uma função linear. Assim, o conjunto é fechado sob multiplicação por escalar.
\end{enumerate}

Portanto, o conjunto de todas as funções lineares $f(x) = mx + b$ com $m, b \in \mathbb{R}$ forma um subespaço vetorial de $V$.

\noindent\textbf{Exemplo 06:} No espaço das matrizes reais $2 \times 2$, $M_{(2,2)}$, considere o conjunto de todas as matrizes simétricas, ou seja, aquelas em que $A = A^T$. Esse conjunto forma um subespaço vetorial de $M_{(2,2)}$. Você pode demonstrar isso verificando as propriedades de um subespaço vetorial

Para tal, é imperativo investigar as três propriedades basilares:

\begin{enumerate}
    \item \textbf{Presença da Matriz Nula:} A matriz nula em $M_{(2,2)}$ é a matriz $\begin{pmatrix} 0 & 0 \\ 0 & 0 \end{pmatrix}$. Nota-se que esta matriz é simétrica, posto que $A = A^T$. Portanto, a matriz nula está asseguradamente contida no conjunto em questão.
    
    \item \textbf{Fechamento sob Adição:} Considerando duas matrizes simétricas $A$ e $B$, a sua soma $A + B$ é também simétrica, visto que $(A + B)^T = A^T + B^T = A + B$. Logo, o conjunto demonstra ser fechado sob adição.
    
    \item \textbf{Fechamento sob Multiplicação por Escalar:} Para qualquer escalar $c$ e matriz simétrica $A$, a multiplicação por escalar $cA$ é igualmente simétrica, haja vista que $(cA)^T = cA^T = cA$. Deste modo, o conjunto revela-se fechado sob multiplicação por escalar.
\end{enumerate}

Assim sendo, constata-se que o conjunto de todas as matrizes simétricas configura-se como um subespaço vetorial de $M_{(2,2)}$.

\section{Combinação Linear}
Dentro de um espaço vetorial, conforme demonstrado que podemos ter subconjuntos de espaços vetoriais, é possível a obtenção de novos vetores a partir de vetores dados \cite{boldrini1980}.

\noindent\textbf{Definição 03:} Sejam $V$ um espaço vetorial $\mathbb{R}$, $v_1, v_2, \ldots, v_n \in V$ e $a_1, \ldots, a_n \in \mathbb{R}$. Então, o vetor $v = a_1v_1 + a_2v_2 + \ldots + a_nv_n$ é um elemento de $V$ podendo ser chamado combinação linear de $v_1, \ldots, v_n$.

Se $V \subset W$, podemos adotar a notação $W = [v_1, \ldots, v_n]$, onde expandindo-o

\centerline{$W = [v_1, \ldots, v_n] = \{v \in V; v = a_1v_1 + \ldots + a_nv_n, a_i \in \mathbb{R}, 1 \leqslant i \leqslant n\}$}

\noindent\textbf{Exemplo 07:} Presuma um vetor $V = \mathbb{R}^3, v \in V, v \neq 0$. Se imaginarmos sua reta que contém o vetor $v$, onde, $[v] = {av: a \in \mathbb{R}}$

\begin{figure}[H]
	\centering
	\includegraphics[scale=0.90]{cb_exemplo7.png}
	\caption{Retirado de \cite{boldrini1980}, pg. 113.}
\end{figure}

\noindent\textbf{Exemplo 08:} Se obtemos $v_1, v_2 \in \mathbb{R}^3$ e $v_3 \in [v_1, v_2]$, então $[v_1, v_2, v_3] = [v_1, v_2]$, então $v_3$ é um combinação linear de  $v_1$ e $v_2$.

\begin{figure}[H]
	\centering
	\includegraphics[scale=0.90]{cb_exemplo8.png}
	\caption{Retirado de \cite{boldrini1980}, pg. 113.}
\end{figure}

\noindent\textbf{Exemplo 09:} Consideremos o espaço vetorial $\mathbb{R}^3$ e os vetores $\mathbf{v} = \begin{pmatrix} 2 \\ 3 \\ 1 \end{pmatrix}$ e $\mathbf{w} = \begin{pmatrix} 1 \\ -1 \\ 2 \end{pmatrix}$. Sejam também os escalares $a = 3$ e $b = -1$. Então temos, os seguintes elementos.

\begin{table}[h]
	\centering
	\begin{tabular}{@{}ccc@{}}
		\toprule
		\textbf{Vetor} & \textbf{Componentes} & \textbf{Escalar} \\ \midrule
		$\mathbf{v}$   & $ 2, 3, 1$   & $3$              \\
		$\mathbf{w}$   & $ 1, -1, 2$  & $-1$             \\ \bottomrule
	\end{tabular}
	\caption{Vetores e escalares utilizados na combinação linear}
\end{table}

Definimos a combinação linear dos vetores $\mathbf{v}$ e $\mathbf{w}$ como:

\[
a\mathbf{v} + b\mathbf{w} = 3 \begin{pmatrix} 2 \\ 3 \\ 1 \end{pmatrix} + (-1) \begin{pmatrix} 1 \\ -1 \\ 2 \end{pmatrix}.
\]

Aplicando as operações, obtemos:

\[
a\mathbf{v} + b\mathbf{w} = \begin{pmatrix} 6 \\ 9 \\ 3 \end{pmatrix} + \begin{pmatrix} -1 \\ 1 \\ -2 \end{pmatrix} = \begin{pmatrix} 6 - 1 \\ 9 + 1 \\ 3 - 2 \end{pmatrix} = \begin{pmatrix} 5 \\ 10 \\ 1 \end{pmatrix}.
\]

Portanto, a combinação linear dos vetores $\mathbf{v}$ e $\mathbf{w}$ com os coeficientes $a = 3$ e $b = -1$ é o vetor 

\centerline{$\begin{pmatrix} 5 \\ 10 \\ 1 \end{pmatrix}$}

\section{Dependência e Independência Linear}
Dado a combinação linear, devemos saber, a priori, se algum desses vetores não é combinação linear dos outros e assim por diante. Para isto precisamos saber sua dependência e independência linear.\nocite{camargo2005}

\noindent\textbf{Definição 03:} Sejam $\mathbf{V}$ um espaço vetorial e $\mathbf{v}_1, \ldots, \mathbf{v}_n \in \mathbf{V}$. Dizemos que o conjunto ${\mathbf{v}_1, \ldots, \mathbf{v}_n}$ é linearmente independente (\textbf{LI}), ou que os vetores $\mathbf{v}_1, \ldots, \mathbf{v}_n$ são \textbf{LI}, se a equação

\centerline{$a_1\mathbf{v}_1 + \ldots + a_n\mathbf{v}_n = 0$}

\noindent implica que $a_1 = a_2 = \ldots = a_n = 0$. No caso em que exista algum $a_i \neq 0$ dizemos que ${v_1, \ldots, v_n}$ é linearmente dependente (\textbf{LD}), ou que os vetores $\mathbf{v}_1, \ldots, \mathbf{v}_n$ são \textbf{LD}.

\noindent\textbf{Teorema 02:} Uma combinação linear é \textbf{LD} se, e somente se um destes vetores for uma combinação linear dos outros.

\centerline{$\{\mathbf{v}_1, \ldots,  \mathbf{v}_n \} =$ \textbf{LD} $\iff \exists i \mid \sum_{i \neq j} c_i \mathbf{v}_i$}

\noindent\textbf{Demonstração:} Sejam $\mathbf{v}_1, \ldots, \mathbf{v}_n$ \textbf{LD} e $a_1\mathbf{v}_1 + \ldots + a_j\mathbf{v}_j + \ldots + a_n\mathbf{v}_n = 0$

Um dos coeficientes deve ser diferente de zero. Suponhamos que seja $a_j \neq 0$. Então 

\centerline{$\mathbf{v}_j = -\frac{1}{a_j}(a_1\mathbf{v}_1 + \ldots + a_{j - 1}\mathbf{v}_{j - 1} + + a_{j + 1}\mathbf{v}_{j + 1} + \ldots + a_n\mathbf{v)_n}$}

\centerline{ e portanto $\mathbf{v}_j = -\frac{a_1}{a_j}\mathbf{v}_1 + \ldots -\frac{a_n}{a_j}\mathbf{v}_n$}

Logo, $\mathbf{v}_j$ é uma combinação linear dos outros vetores.

\noindent\textbf{Exemplo 10:} Sejam $\mathbf{V} = \mathbb{R}^3$ e $v_1, v_2 \in \mathbf{V}$, $\{\mathbf{v}_1, \mathbf{v}_2\}$ é \textbf{LD} $\iff \mathbf{v}_1$ e $\mathbf{v}_2$ estiverem na mesma reta, que passa pela origem. $(\mathbf{v}_1 = \lambda\mathbf{v}_2)$.

\begin{figure}[H]
	\centering
	\includegraphics[scale=0.90]{cb_exemplo10.png}
	\caption{Retirado de \cite{boldrini1980}, pg. 115.}
\end{figure}

% _{m \times n}
% subinscrito
% _
% expoente com mais de um dígito
% x^{21}
