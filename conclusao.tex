\chapter{Considerações Finais}
No decorrer deste trabalho enfatizamos a relevância do estudo da álgebra linear, destacando as transformações lineares e suas aplicações em diversas áreas, desde a matemática pura até a engenharia. As transformações lineares, em função de suas inúmeras aplicações e de sua versatilidade, é uma ferramenta matemática poderosa e seu estudo é fundamental por ser uma disciplina altamente aplicável, tanto no campo teórico, quanto no mundo real. E é está importância e aplicabilidade que justificou a escolha deste tema para este trabalho.

Abordamos as definições e conceitos fundamentais relacionados a álgebra linear e a transformação linear, para assim proporcionar a base necessária ao entendimento do tema. Dentre os assuntos fundamentais ao entendimento da álgebra linear, tratamos dos espaços e subespaços vetoriais, combinação linear, dependência e independência linear, base e dimensão.

Ao estudar especificamente as transformações lineares consideramos o domínio e o contradomínio como espaços vetoriais reais, ou seja, são funções vetoriais. Nesta etapa do trabalho passamos pelo estudo do núcleo de uma transformação linear e pelo conceito de isomorfismo, sendo que eles estão relacionados pelo Teorema do núcleo e da imagem, o qual estabelece uma relação entre a dimensão do núcleo, a dimensão da imagem e a dimensão do domínio.

Ao longo do desenvolvimento deste trabalho encontramos diversas aplicações para transformações lineares, como, por exemplo, no processamento de sinais, análise de dados, computação gráfica, sistemas dinâmicos de engenharia elétrica, dentre outros. Em nosso trabalho enfatizamos a aplicação de transformações lineares no posicionamento de um braço robótico, aplicações na área da educação, na criptografia e classificação de imagens em rede neural. Em relação ao braço robotizado o estudo se baseou nos movimentos de rotação e translação em relação a um referencial e sua consequente mudança de posição ao aplicar a transformação linear. Foram utilizadas representações gráficas desses movimentos de rotação e translação, associadas a transformação linear, o que ilustrou e facilitou o entendimento da aplicação real desta poderosa  ferramenta matemática.

No campo da educação a transformação linear pode ser utilizada, com ou sem recursos tecnológicos (software específicos, com o GeoGebra), para ajudar no desenvolvimento do pensamento matemático, necessário a todos os alunos, e sobretudo, naqueles que seguirem as áreas de ciências exatas, portanto, seu estudo é altamente relevante. As Transformações lineares ainda tem grande destaque na criptografia, uma vez que pode tornar a troca de mensagens e o armazenamento de dados mais seguros e confiáveis.

Por fim, uma aplicação onde a transformação linear tem ganhado empregabilidade é o de classificação de imagens utilizando redes neurais,  uma vez que as transformações lineares auxiliam na resolução de problemas de classificação.
Como foi exposto, a transformação linear tem diversos empregos nos mais diversos ramos das ciências, sendo um recurso de grande importância para estudantes e para profissionais, como matemáticos, físicos e engenheiros, seja para o desenvolvimento do pensamento matemático, seja para aplicações práticas e reais.

Com o rápido desenvolvimento tecnológico atual, cada vez mais a álgebra linear e as transformações lineares encontrarão novas aplicações e novos estudos serão necessários para intensificar ainda mais sua aplicabilidade.

