\chapter{Revisão da Literatura}
Começaremos pela definição de um espaço vetorial e seu subespaço de um número $\mathbb{R}$:

\textbf{Definição 01:} Seja um conjunto V, não vazio, com duas operações: soma, $V \times V \rightarrow V$, e multiplicação por escalar, $R \times V \rightarrow V$, tais que, para quaisquer $u, v, w \in \mathbb{R}$, lembrando que as propriedades $(u + v) + w = u + (v + w)$ e $1u = u$.

\textbf{Observação:} Limitaremos nossa discussão, demonstrações e aplicações dentro do conjuntos dos números reais apenas.

\textbf{Definição 02:} Dado um espaço vetorial V, um subconjunto W, não vazio, será um subespaço vetorial de V se:
\begin{enumerate}
	\item Para quaisquer $u, v \in W$ tivermos $u + v \in W$.
	\item Para quaisquer $a \in R, u \in W$ tivermos $au \in W$.
	\end{enumerate}

Sabendo tais definições, podemos expressar agora a definição de um transformação linear:
\textbf{Definição 03:} Sejam V e W dois espaços vetoriais. Uma transformação linear (aplicação linear) é uma função de V em W, $F:V \rightarrow W$, que satisfaz as seguintes condições:
\begin{enumerate}
	\item Para quaisquer u e v em V, $F(u + v) = F(u) + F(v)$.
	\item Para quaisquer $k \in R e v \in V$, $F(kv) = kF(v)$.
\end{enumerate}	

Não se esqueça de demonstrar e provar tudo isso aí $\ldots$