\chapter{Introdução}
Uma área da Matemática que tem implicações na computação gráfica, genética, criptografia, redes elétricas entre outros é a Álgebra Linear (AL). Com estrutura que permite um tratamento algébrico simples, a AL estuda os aspectos relacionados ao Espaço Vetorial (EV). Um conceito central da AL é a Transformação Linear (TL), que desempenham papel fundamental na análise e compreensão dos sistemas lineares de equações, geometria analítica, física, engenharia e outros campos de estudo \cite{figueiredo2009}. 

Contextualizando o início dos estudos da AL, que é o estudo dos espaços vetoriais e das TL entre eles e possui variadas aplicações \cite{souzasilzaeliza2017}, nos meados do século XVIII, Euler e Louis Lagrange publicaram o "Recherche d'Arithmétique", entre 1773 e 1775, no qual estudavam certos conceitos da TL. Posteriormente, Johann Carl Friedrich Gauss, também estudou sobre assuntos que apresentou similaridade com a matriz de transformação linear.

No século XIX e XX, Giuseppe Peano cunha o termo "sistema linear" com a primeira definição de axiomática para espaço vetorial. Atualmente, a apresentação da AL, temas abordados nesse campo da matemática são frequentemente esquecidos. Este estudo busca o entendimento e compreender sobre as transformações lineares em sua totalidade e aplicações no contexto atual contemporâneo.

Passo esse brevíssimo contexto histórico e motivador para a nossa pesquisa e deleite desramo de estudado, iremos nos adiantar a certos conceitos matemáticos elementares já bastantes fundamentados no decorrer dos anos escolares do ensino básico regular. para isto, passaremos a certas definições matemáticas primordiais que serão apresentadas nesta monografia para as discussões advindas a posteriori neste estudo.

Portanto, dividimos esta monografia em 4 capítulos: revisão literária fundamentais, pesquisas de artigos, teses e discussões recentes sobre as transformações lineares em diversas aplicações, seu contexto educacional atual em questão de matéria aplicada e por conseguinte nossa metodologia utilizada, os resultados obtidos dessa pesquisa e, por fim, nossa discussão final, a saber, do uso da transformação linear atualmente.

\section{Justificativa}
As transformações lineares se fazem presentes em diversos campos da matemática, e sua aplicação é fundamental para a solidificar a base teórica de problemas práticos. A partir da compreensão de conceitos e das propriedades das TL, a modelagem e a solução de problemas complexos são facilitadas, como na tecnologia e computação, por exemplo. 

Procura-se contribuir com o raciocínio lógico e a capacidade de abstração, necessários para o desenvolvimento das habilidades matemáticas e analíticas. Essa investigação visa contribuir com o avanço do conhecimento nessa área e fundamentar o desenvolvimento de novos métodos, teorias e aplicações. 

\subsection{Objetivo Geral}
Este trabalho objetiva compreender a aplicação da transformação linear.

\subsection{Objetivos Específicos}
Como objetivos específicos, são apresentados:

\begin{itemize}
	\item Uso da transformação linear na sociedade;
	\item Aplicação em modelagem matemática e em contexto computacional da transformação linear;
	\item Contextualização da transformação linear no campo da inteligência artificial. 
\end{itemize}
