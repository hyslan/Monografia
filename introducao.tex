\chapter{Introdução}
Com o decorrer do tempo, depois da era de ouro da álgebra linear nos meados do século XVIII. Onde, Euler e Louis Lagrange publicaram o "Recherche d'Arithmétique", entre 1773 e 1775, no qual estudavam certos conceitos da transformação linear. Posteriormente, Johann Carl Friedrich Gauss, também estudou sobre assuntos que apresentou similaridade com a matriz de transformação linear.

Até se arrefecer o assunto no século XIX e XX, com Giuseppe Peano, onde foi cunhado o termo "sistema linear" com a primeira definição de axiomática para espaço vetorial. Nos dias atuais, a apresentação da álgebra linear, temas abordados nesse campo da matemática são frequentemente esquecidos, portanto, este estudo trata de buscar o entendimento e compreender sobre as transformações lineares em sua totalidade e aplicações no contexto atual contemporâneo.

Passado esse brevíssimo contexto histórico e motivador para a nossa pesquisa e deleite deste ramo de estudado, iremos nos adiantar a certos conceitos matemáticos elementares já bastantes fundamentados no decorrer dos anos escolares do ensino básico regular. Para isto, passaremos a certas definições matemáticas primordiais que serão apresentadas nesta monografia para as discussões advindas a posteriori neste estudo.

Portanto, dividimos esta monografia em 4 capítulos, a saber, revisão literária fundamentais, pesquisas de artigos, teses e discussões recentes sobre as transformações lineares em diversas aplicações, seu contexto educacional atual em questão de matéria aplicada e por conseguinte nossa metodologia utilizada, os resultados obtidos dessa pesquisa e, por fim, nossa discussão final, a saber, do uso da transformação linear atualmente.