\chapter{Introdução}
Uma área da Matemática que tem implicações na computação gráfica, genética, criptografica, redes elétricase outros é a Álgebra Linear (AL). Com estrutura que permite um tratamento algébrico simples, a AL estuda os aspectos relacionados ao Espaço Vetorial (EV). Um conceito central da AL é a Transformação Linear (TL), que desempenham papel fundamental na análise e compreensão dos sistemas lineares de equações, geometria analítica, física, engenharia e outros campos de estudo. % Citar Rios,Figueiredo e Cunha, 2009

Contextualizando o início dos estudos da AL, que é o estudo dos espaços vetoriais e das TL entre eles e possui variadas aplicações % citar Souza SIlva e Costa da SIlva, 2017
, nos meados do século XVIII, Euler e Louis Lagrange publicaram o "Recherche d'Arithmétique", entre 1773 e 1775, no qual estudavam certos conceitos da TL. Posteriormente, Johann Carl Friedrich Gauss, também estudou sobre assuntos que apresentou similaridade com a matriz de transformação linear.

No século XIX e XX, Giuseppe Peano cunha o termo "sistema linear" com a primeira definição de axiomática para espaço vetorial. Nos dias atuais, a apresentação da AL, temas abordados nesse campo da matemática são frequentemente esquecidos. Este estudo busca o entendimento e compreender sobre as transformações lineares em sua totalidade e aplicações no contexto atual contemporâneo.

Passo esse brevíssimo contexto histório e motivador para a nossa pesquisa e deleite desramo de estudado, iremos nos adiantar a certos conceitos matemáticos elementares já bastantes fundamentados no decorrer dos anos escolares do ensino básico regular. para isto, passaremos a certas definições matemáticas primordiais que serão apresentadas nesta monografia para as discussões advindas a posteriori neste estudo.

Portanto, divimos esta monografia em 4 capítulos: revisão literária fundamentais, pesquisas de artigos, teses e discussões recentes sobre as transformações lineares em diversas aplicações, seu contexto educacional atual em questão de matéria aplicada e por conseguinte...
