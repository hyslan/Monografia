\documentclass[
12pt,
 a4paper,
 % chapter=TITLE,
  % section=TITLE,
   % subsection=TITLE,
    % subsubsection=TITLE,
    brazil,
    % english,
    openright,
    twoside
    ]{abntex2}
    
% Pacotes
% ---
\usepackage[brazilian,hyperpageref]{backref}
\usepackage[alf]{abntex2cite}
\usepackage[utf8]{inputenc}
\usepackage[T1]{fontenc}
\usepackage{amsmath, amsfonts, amssymb}
\usepackage{float}
\usepackage{graphicx}
\usepackage{indentfirst}
\usepackage{hyperref}
\usepackage{geometry}
\usepackage{times}
\usepackage{setspace}
\usepackage{microtype}
\usepackage{color} 
% ---


% Configurações de margens
\geometry{
	a4paper,
	left=3cm,
	right=2cm,
	top=3cm,
	bottom=2cm
}

% Dados do TCC/Monografia
\titulo{Transformações Lineares e suas aplicações}
\autor{Hyslan Silva Cruz\\Iara Regina Grilo Papais}
\orientador[Orientadora:]{Lorena Salvi Stringheta}
\instituicao{Universidade Virtual do Estado de São Paulo} 
\local{Suzano}
\data{2024}
\tipotrabalho{TCC}
\preambulo{Monografia de graduação à Universidade Virtual do Estado de São Paulo, como requisito parcial para a obtenção do título de Licenciatura em Matemática.
	
	Orientadora: \imprimirorientador}


% Configurações de aparência do PDF final

% alterando o aspecto da cor azul
\definecolor{blue}{RGB}{41,5,195}

% --- 
% Espaçamentos entre linhas e parágrafos 
% --- 

% O tamanho do parágrafo é dado por:
\setlength{\parindent}{1.3cm}

% Controle do espaçamento entre um parágrafo e outro:
\setlength{\parskip}{0.2cm}  % tente também \onelineskip

% Início do Documento
\begin{document}
	
	\selectlanguage{brazil}
	
	% Retira espaço extra obsoleto entre as frases.
	\frenchspacing 
	
	% Capa
	\imprimircapa
	
	% Folha de Rosto
	\imprimirfolhaderosto
	
	% Resumo
	% resumo em português
	\setlength{\absparsep}{18pt} % ajusta o espaçamento dos parágrafos do resumo
	\begin{resumo}
		Resumo de nosso trabalho.
		
		\textbf{Palavras-chave: Transformação Linear, Álgebra Linear, Matrizes}
	\end{resumo}
		
	
	% Abstract
	% resumo em inglês
	\begin{resumo}[Abstract]
		\begin{otherlanguage*}{english}
			This is the english abstract.
			
			\vspace{\onelineskip}
			
			\noindent 
			\textbf{Keywords}: latex. abntex. text editoration.
		\end{otherlanguage*}
	\end{resumo}
	
	% inserir lista de tabelas
	% ---
	\pdfbookmark[0]{\listtablename}{lot}
	\listoftables*
	\cleardoublepage
	% ---
	
	% Lista de símbolos
	\begin{simbolos}
		\item[$ \mathbb{R} $] Conjunto dos números reais.
		\item[$\exists$] Símbolo de existe.
	\end{simbolos}
	
	
	% Sumário
	\pdfbookmark[0]{\contentsname}{toc}
	\tableofcontents*
	\cleardoublepage
	
	
	% Capítulos
	\textual
	
	\chapter{Introdução}
Uma área da Matemática que tem implicações na computação gráfica, genética, criptografica, redes elétricase outros é a Álgebra Linear (AL). Com estrutura que permite um tratamento algébrico simples, a AL estuda os aspectos relacionados ao Espaço Vetorial (EV). Um conceito central da AL é a Transformação Linear (TL), que desempenham papel fundamental na análise e compreensão dos sistemas lineares de equações, geometria analítica, física, engenharia e outros campos de estudo. % Citar Rios,Figueiredo e Cunha, 2009

Contextualizando o início dos estudos da AL, que é o estudo dos espaços vetoriais e das TL entre eles e possui variadas aplicações % citar Souza SIlva e Costa da SIlva, 2017
, nos meados do século XVIII, Euler e Louis Lagrange publicaram o "Recherche d'Arithmétique", entre 1773 e 1775, no qual estudavam certos conceitos da TL. Posteriormente, Johann Carl Friedrich Gauss, também estudou sobre assuntos que apresentou similaridade com a matriz de transformação linear.

No século XIX e XX, Giuseppe Peano cunha o termo "sistema linear" com a primeira definição de axiomática para espaço vetorial. Nos dias atuais, a apresentação da AL, temas abordados nesse campo da matemática são frequentemente esquecidos. Este estudo busca o entendimento e compreender sobre as transformações lineares em sua totalidade e aplicações no contexto atual contemporâneo.

Passo esse brevíssimo contexto histório e motivador para a nossa pesquisa e deleite desramo de estudado, iremos nos adiantar a certos conceitos matemáticos elementares já bastantes fundamentados no decorrer dos anos escolares do ensino básico regular. para isto, passaremos a certas definições matemáticas primordiais que serão apresentadas nesta monografia para as discussões advindas a posteriori neste estudo.

Portanto, divimos esta monografia em 4 capítulos: revisão literária fundamentais, pesquisas de artigos, teses e discussões recentes sobre as transformações lineares em diversas aplicações, seu contexto educacional atual em questão de matéria aplicada e por conseguinte...

		
	\chapter{Revisão da Literatura}
A literatura principal utilizada foi Álgebra Linear de Boldrini (1980).
	
	\chapter{Metodologia}
Estudo sobre a literatura e aplicação direta, principalmente computacional.
	
	\chapter{Resultados}
Os resultados foram...
	
	% Conclusão
	\chapter{Conclusão}
That's all folks!
Teste citação: De acordo com \cite{einstein1905}, a teoria da relatividade restrita foi publicada por Einstein em 1905.

		
	% Referências Bibliográficas
	\bibliographystyle{abntex2-alf}
	\bibliography{references.bib}
		
\end{document}