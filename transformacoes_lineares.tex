\chapter{Transformações Lineares}
Neste capítulo, iremos tratar sobre um tipo especial de função ou aplicação, onde, segundo \cite{steinbruch1987}, o domínio e o contradomínio são espaços vetoriais reais. Assim, tanto a variável independente como a variável dependente são vetores, razão pela qual essas funções são chamadas vetoriais.



\noindent\textbf{Definição 03:} Sejam V e W dois espaços vetoriais. Uma transformação linear (aplicação linear) é uma função de V em W, $F:V \rightarrow W$, que satisfaz as seguintes condições:
\begin{enumerate}
	\item Para quaisquer $u$ e $v$ em $V$, $F(u + v) = F(u) + F(v)$.
	\item Para quaisquer $k \in R$ e $v \in V$, $F(kv) = kF(v)$.
\end{enumerate}	